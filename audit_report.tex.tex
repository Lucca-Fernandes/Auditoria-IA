\documentclass{memoir}%
\usepackage[T1]{fontenc}%
\usepackage[utf8]{inputenc}%
\usepackage{lmodern}%
\usepackage{textcomp}%
\usepackage{lastpage}%
\usepackage{geometry}%
\geometry{tmargin=1cm,lmargin=2cm,rmargin=2cm}%
\usepackage[brazilian]{babel}%
\usepackage{titlesec}%
\usepackage{fancyhdr}%
\usepackage{xcolor}%
\usepackage{noto}%
\usepackage{hyperref}%
%
\title{Relatório de Avaliação de Aula}%
\author{Sistema de Auditoria de Aulas (IA)}%
\date{\today}%
%
\begin{document}%
\normalsize%
\hypersetup{colorlinks=true,linkcolor=blue,urlcolor=blue,citecolor=blue,pdfpagemode=UseNone,pdfstartview=FitH}%
\maketitle%
\section{Resultados da Avaliação}%
\label{sec:ResultadosdaAvaliao}%
Nota Final: 68.33%
Decisão: \textbf{Aprovada}%
\vspace{0.5cm}%
\subsection*{Qualidade técnica do conteúdo (Nota: 70)}%
\textbf{Justificativa:} O conteúdo aborda a integração de uma API em um aplicativo Flutter, mostrando passos relevantes.  Há algumas imprecisões na descrição de alguns passos (ex: 'flater por hoje'), mas o conceito geral é transmitido. A demonstração prática é parcialmente bem sucedida, apesar da falta de clareza em alguns momentos.  A explicação do processo de integração e uso da API é razoavelmente completa.%
\par%
\textbf{Sugestão:} Rever o conteúdo para eliminar imprecisões e garantir que todos os passos sejam descritos de forma clara e concisa. Criar uma estrutura mais organizada para a demonstração prática.%
\par%
\vspace{0.3cm}%
\subsection*{Linguagem corporal (Nota: 70)}%
\textbf{Justificativa:} A pontuação de movimento de 4.096731085526317 corresponde a um nível de engajamento razoável, traduzido para uma pontuação aproximada de 70.  Não há informações na transcrição para avaliar a postura, contato visual e expressividade diretamente, porém a suposição é de um bom nível baseado no dado fornecido.%
\par%
\textbf{Sugestão:} Embora o movimento seja considerado bom, observe se a linguagem corporal auxilia na transmissão do conteúdo. Pratique gestos que complementem a explicação, mantendo um contato visual constante com a câmera (ou a audiência).%
\par%
\vspace{0.3cm}%
\subsection*{Tom de voz (Nota: 60)}%
\textbf{Justificativa:} A transcrição revela vícios de linguagem frequentes ('né', 'aí', 'ok'),  falta de clareza em algumas frases e um ritmo irregular na fala. A modulação e entonação parecem inconsistentes, prejudicando a compreensão em alguns trechos.%
\par%
\textbf{Sugestão:} Trabalhar na dicção para eliminar vícios de linguagem.  Variar a entonação e o ritmo da fala para tornar a aula mais envolvente e fácil de acompanhar. Usar pausas estratégicas.%
\par%
\vspace{0.3cm}%
\subsection*{Clareza e estrutura do roteiro (Nota: 75)}%
\textbf{Justificativa:} A aula apresenta uma estrutura básica (introdução, desenvolvimento, conclusão), mas a organização poderia ser melhorada. A sequência de passos na demonstração prática às vezes é confusa,  faltando clareza em algumas transições entre etapas. A introdução do conceito é razoavelmente boa.%
\par%
\textbf{Sugestão:} Reestruturar o roteiro para tornar a sequência de passos mais clara e lógica.  Incluir uma introdução mais concisa e uma conclusão mais abrangente.  Utilizar tópicos e subtópicos para melhor organização.%
\par%
\vspace{0.3cm}%
\subsection*{Ritmo da apresentação (Nota: 65)}%
\textbf{Justificativa:} O ritmo da apresentação é irregular, com momentos de fala rápida e confusa, intercalados com pausas abruptas.  A falta de pausas estratégicas para permitir a absorção do conteúdo prejudica a fluidez geral.%
\par%
\textbf{Sugestão:} Regular o ritmo da fala, incluindo pausas para respiração e compreensão do conteúdo pelo aluno.  Gravar a aula e analisá-la para identificar e corrigir pontos de aceleração ou lentidão.%
\par%
\vspace{0.3cm}%
\subsection*{Didática (Nota: 70)}%
\textbf{Justificativa:} A didática demonstra pontos positivos e negativos. O professor tenta explicar conceitos, mas a falta de clareza em alguns pontos e o excesso de detalhes desnecessários dificultam o aprendizado. O uso de exemplos é presente, mas poderia ser mais eficaz e direcionado.%
\par%
\textbf{Sugestão:} Utilizar analogias mais eficazes e exemplos mais concisos e direcionados ao tema.  Simplificar a linguagem e evitar jargões técnicos desnecessários. Criar materiais de apoio para facilitar o aprendizado.%
\par%
\vspace{0.3cm}%
\subsection*{Qualidade geral (Nota: 68)}%
\textbf{Justificativa:} A aula apresenta um conteúdo relevante, mas a entrega é prejudicada pela falta de clareza na fala, ritmo irregular e organização do roteiro.  Melhorias na dicção, organização e ritmo são cruciais para uma melhor experiência de aprendizado.%
\par%
\textbf{Sugestão:} Rever todos os aspectos da aula, considerando as sugestões anteriores, para melhorar a experiência de aprendizado do aluno.  Testar a aula com um público-alvo para coletar feedback e realizar ajustes necessários.%
\par%
\vspace{0.3cm}

%
\end{document}